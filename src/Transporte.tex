\chapter{El problema del Transporte}
	\section{Cálculo de la solución inicial}
	
	%Pendiente
	%\subsection{Método de la esquina noroeste}
	%\subsection{Método de costo mínimo}
	
	\subsection{Método de Voguel}
	
	\begin{enumerate}
		\item Construir la matriz de transporte con las fábricas en la columnna de la izquierda y las bodegas en las columnas restantes. Se agrega a cada celda el costo de envío de cada fábrica a cada bodega.
		
		\item Verificar si el problema está balanceaado. Si la suma de las capacidades de las bodegas y las fábricas no es igual, agregamos una fila o una columna según donde haga falta, y su capacidaad será la cantidad faltante.
		
		\item Si debemos minimizar calculamos las diferencias entre los elementos más pequeños de cada renglóń y cada columna; si debemos maximizar calculamos las diferencias entre los elementos más grandes en cada renglón y en cada columna.
		
		\item Seleccionamos el renglón o columna con la diferencia más grande, si existe un empate se selecciona el reglón que contenga el costo menor (en caso de minimizar) o el costo mayor (en caso de maximizar). Si el empate persiste se selecciona la celda que pueda recibir más unidades. Si el empate persiste se selecciona cualquiera de las celdas que empataron.
		
		\item En caso de satisfacer por completo la demanda de una columna o un renglón, se cancelan las celdas que hayan quedado vacías en dicha columna o renglón.
		
		\item Se repiten el paso 3 al 5 hasta que ya no haya celdas que rellenar.
	\end{enumerate}

	\section{Prueba de la degeneración}
	
	Si $R+C-1 < A$, $R$ y $C$ son la cantidad de filas y columnas en la matriz de asignación, y $A$ es la cantidad de asignaciones en la solución, entonces se tiene una \textbf{solución degenerada}. En tal caso es necesario agregar un valor $\mathcal{E}$ (Epsilon) que nos ayude a calcular los índices de mejoramiento para todas las celdas vacías. Se puede comenzar probando la celda vacía con el costo menor.
	
	La degeneración representa una ambigüedad en la solución porque hay más grados de libertad en el problema de transporte de lo que se esperaría. Es decir, hay más posibles formas de asignar unidades de manera eficiente, pero no todas esas formas están representadas explícitamente en la tabla de asignación. La degeneración puede ocurrir cuando hay algunas celdas básicas que contienen una cantidad de cero unidades.
	
	En términos prácticos, la degeneración puede llevar a la dificultad para calcular los índices de mejoramiento en los métodos como el cruce del arroyo o el método MODI, ya que la cantidad de celdas básicas no es suficiente para garantizar que la solución sea bien determinada y que las mejoras puedan ser identificadas de forma directa.

	\section{Prueba de optimalidad}
	
	\subsection{Método del cruce del Arroyo}
	
	Se calculan los \textbf{índices de mejoramiento} para todas las celdas \textbf{celdas no básicas} (vacías). 
	Comenzando a partir de cada celda no básica se intenta construir una trayectoria cerrada saltando horizontalmente y verticalmente hacia otras \textbf{celdas básicas} (con asignación), no se permiten saltos en diagonal. El índice de mejoramiento de cada celda se calcula sumando el costo de cada celda de la trayectoria construida alternando signos a partir de la celda vacía donde se comienza. La celda de partida siempre tiene signo positivo.
	
	\subsection{Método de Modi}
	
	\begin{itemize}
		\item Se calculan los valores de $R_{i}$ y $K_{j}$ para cada renglón y cada columna.
		\item Siempre se comienza con $R_ {1}=0$.
		\item Se calcula el resto de los valores mediante $R_{i}+K_{j}=C{ij}$
	\end{itemize}
	
	Si todos los índices de mejoramiento son positivos o cero (en caso de minimización), entonces hemos llegado a la solución, pero si existe por lo menos un negativo, debemos mover unidades; para el caso de maximización, si todos los valores son negativos o cero, significa que hemos llegado a la solución, pero si hay por lo menos un índice positivo significa que debemos mover unidades.
	
	\section{Reasignación de unidades \small{(Caso especial, mejorando la solución)}}
	
	En el caso de minimización, se selecciona la \textbf{celda no básica} (una celda sin asignación) que tenga el índice de mejoramiento más negativo, esta será nuestra nueva \textbf{celda básica} a la cual se moverán unidades de asignación y se incorporará a nuestra solución. Para el caso de maximización, se elige la celda con el índice de mejoramiento más positivo.
	
	Una vez seleccionada esta celda, se traza un ciclo cerrado siguiendo el método del cruce del arroyo. En este ciclo, las celdas alternan signos, comenzando con la celda elegida como positiva.
	
	A continuación, se identifica la celda con signo negativo que tenga la asignación más pequeña. Esta celda se convertirá en un celda no básica. Luego, se actualizan las asignaciones dentro del ciclo cerrado:
	
	\begin{itemize}
		\item Se resta el valor de la asignación de la celda seleccionada a todas las celdas con signo negativo.
		\item Se suma ese mismo valor a todas las celdas con signo positivo.	
	\end{itemize}
	
	
	Este proceso permite mejorar la solución mientras se mantiene el balance de oferta y demanda.
	
	Si la solución obtenida previamente es degenerada y una celda perteneciente a la trayectoria trazada contiene el valor $\epsilon$ y un signo negativo no será posible mejorar la solución, por lo tanto debemos buscar otra celda para asignar el valor $\epsilon$ y volver a calcular los índices de mejoramiento.
	
	
	%Pendiente
	%\section{Ejercicios propuestos}