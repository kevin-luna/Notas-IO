	\chapter{El problema de asignación}
	Los problemas de asignación forman una subclase del problema de transporte.
	
	\section{El método Húngaro}
	
	\begin{enumerate}
		\item Elaborar la matriz de costos de oportunidad como si de un problema de transporte se tratase.
		\item Verificar y manejar los casos especiales.
		\item Una vez manejados los casos especiales, se procede a buscar el elemento más pequeño de cada fila y se resta de cada elemento de la fila.
		\item Se repite el paso anterior para los elementos de cada columna.
		\item Determine si la solución es óptima intentando cubrir todos los ceros de la matriz utilizando $\max(R,C)$ líneas rectas, si esto es posible hemos llegado a la solución óptima.
		
		\item Si no se ha llegado a la solución óptima buscamos el elemento más pequeño de los no cubiertos por alguna línea, se suma a cada número que se encuentre en la intersección de dos líneas y se resta de todos los números no cubiertos por alguna línea. El resto de los números se mantienen igual. Al terminar, repita el paso anterior.
	\end{enumerate}
	
	\section{Casos especiales}
	
	\subsection{Costos negativos}
	
	Se agrega a cada celda un valor igual que el costo más negativo.
	
	\subsection{Maximización}
	
	Se busca el elemento más grande de la matriz de costos y cada uno de los elementos de la matriz se reemplazan por la diferencia entre dicho valor y el costo en cada celda. Después se procede de manera normal con el método Húngaro.
	
	\subsection{Asignaciones prohibidas}

	Si existen celdas en donde no podamos realizar asignaciones debido a las restricciones del problema, asignamos a la celda un costo más alto que el mayor de toda la matriz (en caso de maximización) o un costo más bajo que el mínimo de toda la matriz (en caso de minimización), así garantizamos que las celdas que los contienen nunca formarán parte de la solución final.
	
	\subsection{Soluciones óptimas múltiples}
	
	Si en el momento de realizar la asignación para la solución final aparecen opciones múltiples de ceros para asignar existen múltiples soluciones.
	
	\subsection{Problema no balanceado}
	
	Se agrega una fila o columna ficticia, según sea necesario, y su capacidad será la cantidad de unidades faltantes.
	